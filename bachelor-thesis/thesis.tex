\documentclass[12pt]{report}

\title{De-ossifying Internet Transport: a case for NEAT}
\author{Helena Gummedal}

\begin{document}

\maketitle

% should set the pages before the table of contents to roman numbers
\chapter*{Acknowledgments}%
%\addcontentsline{toc}{chapter}{\numberline{}Acknowledgments}%
% if you have each of the chapters in it own file, use input to put it in here (under each chapter)
Thanks to people that have helped with the thesis.

\chapter*{Abstract}%
%\addcontentsline{toc}{chapter}{\numberline{}Abstract}%
Summary of the thesis.

\tableofcontents

\chapter{Introduction}
\textit{General information about socket programming, transport protocols and NEAT. How history has evolved - or not evolved. This will then lead us to the purpose of the thesis which will be described along with the objectives of it (which will most likely be in its own section/ under chapter). This chapter may also include work methods, tools, thesis structure and motivation (may then be in their own sections).}  \\
\\
For the past 30 years the Berkeley sockets (BSD sockets) has been the ... At the time ...
\begin{itemize}
\item Introduce BSD sockets (what is used now)
\item What are the problems?
\item Transport layer ossified
\item Only two protocols used, really
\item Why do we have these problems
\item Other protocols available
\item How to solve these problems - several proposals - NEAT the one to be discussed here (TAPS by IETF ...)
\item Brief introduction of NEAT (what it is, how it works and why use it)
\item NEAT has already been evaluated and compared by another master thesis; this thesis will build on this thesis findings
\end{itemize} 

Research problem: how can NEAT be improved? Json handling? Can the overhead of NEAT be improved by substituting the json handling? With bit mask?
\begin{itemize}
\item How network programming and the internet transports protocols work
\item More protocols that aren’t really used. NEAT solve this
\item NEAT - what is it, who developed it, what are the mechanisms, what is NEAT built on
\item Master thesis has evaluated NEAT to see if it can compete with the already established ways of networking programming (epoll (not evaluated since the experiments were done on freebsd), kqueue, libuv). Build on the results of this thesis and try to replicate their results (or just see if they are the same). This thesis will be on linux and not freebsd though. From this master thesis it was concluded that the overhead was large compared to the other available methods and this would need improvement. The json handling used a lot of overhead and there should therefore be looked for a different solution for this problem. Bit mask?
\end{itemize}
This may also be a section in the introduction. Mention when, and why neat was created an by who - general information in that regard. Mention the master thesis and what the purpose of that thesis was. Also what it acheived (its results) \\
\\
TCP and UDP has for many years been the only used transport protocols. Recently more protocols has been established to handle functions and provide services that neither TCP or UDP support fully. Such protocols are SCTP (multistreaming), UDP-lite (...) and .... However, largely due to the inflexibility of the BSD sockets, these transport protocols are rarely used. The NEAT library want to change this situation. It is a callback library developed by ... in ...It uses the happy eyeballs mechanism (source - mention these sources and previous research into it?) which has been suggested before as a solution to this problem. It also uses libuv as an event library. Instead of the current way networking APIs work where a transport protocol is necessary for the establishment of the connection, NEAT wants instead to offer network services such as reliability, multistreaming. The by the use of the happy eyeballs mechanism determined the most suitable protocol. This way the NEAT project wants to then \textit{achieve a complete redesign of the way in which Internet applications interact with the network}(source). 
\\ \\
Mention libuv, kqueue in greater detail? \\ \\
This thesis will be built on the evaluations and findings of an earlier master thesis written at UiO (source). From these results the goal is to improve and optimize the NEAT library. This master thesis goal was the evaluation of the NEAT library by comparing it to other networking APIs. Comparison of CPU time and the memory usage was the focus of the thesis which was done through several experiments on FreeBSD with teaCup. \\
\\
A master's thesis, Performance Evaluation of NEAT Internet Transport Layer API and Library by Fredrik Haugseth, written at UiO has conducted research that compared the NEAT library to other networking APIs such as libuv and kqueue. The thesis focused on the memory overhead and it concluded that optimization was in need for NEAT to be in league with the other networking APIs. A quite higher CPU overhead was largely contributed to the use of json; which was a lot. 
\\
Mention that this thesis is based on a previous published master thesis (this may also just be mentioned in the introduction; depends on the amount of information that will be included). \\
Some of these topics may be better other places.


\section{Objectives / Problem statement}
What is the purpose of the thesis? Replicate the results of tests carried out by a previous master thesis, then optimize the NEAT code (replace json handling since it is the reason for the high overhead - from master thesis) and test NEAT again and hopefully an improvement in the CPU overhead will be seen. Often a question is asked that will be answered in the conclusion. Some possible questions: 
\begin{itemize}
\item What improvements can be made to NEAT?
\item Is it possible to make improvements to neat to make it "competitive" to the socket programming today?
\item Why NEAT?
\item Were we able to solve the problem of the high overhead that NEAT suffers from? How?
\end{itemize}

In this thesis the findings of the previous master thesis ... will be the foundation. The master thesis discovered among other that NEAT, as opposed to the other networking APIs (libuv and kqueue), had much higher CPU usage. It seemed that this was in large part due to the JSON operations related to converting strings to JSON objects and vice versa when opening many NEAT flows (source). As such, in this thesis the objective is to replicate the results of this master thesis and then improve the findings (CPU usage).

\section{Research Questions}
In this thesis the research question to be answered is: 
\begin{itemize}
\item Can NEAT be optimized by the replacing json operations related to string manipulation?
\end{itemize}

\section{Contributions}
\textit{What have I contributed.} \\
Through this bachelor thesis the following have been done:
\begin{enumerate}
\item Replicated some of the results of the master thesis ...
\item Improved / optimized the NEAT library 
\end{enumerate}


\section{Research Methodology}
\textit{How are the experiments run; which will probably be the same way that was done in the master thesis with teacup since it is to be replicated. Then after improvements are made by NEAT, the same experiments will be carried out again. Mention that}  \\ \\
In this thesis the evaluation of the results has been through ...
The tests that have been executed are done in the same fashion as the master thesis to create the best comparable result. This means that ...
Unlike the master thesis, these tests are run on linux (on VirtualBox) and not FreeBSD. As such, minor changes have been done to the testbed that was used in that thesis.
If teacup is used - mention it (raspberry pi too if used)


\chapter{Background / Literature review}

Introduce all the relevant theory for this thesis.

\section{Transport Layer}
General about the layers (short), then about the transport layer, the available transport protocols (TCP and UDP; also the newer and less used like SCTP, UDP-lite ...) - some information about them. \\
May have a diffrent title and then have transport layer as an under section - will see. Can then mention the difference models and layers in them shortly before moving on to the transposrt layer and the transport protocols.

\subsection{TCP}
Transmission control protocol (TCP) is a transport protocol that provides reliable delivery of the data which is sent through ... Its responsibilities include the numbering and tracking of data segments, the acknowledgements of received data segments and the retransmission of any unacknowledged packages after a certain amount of time. TCP also provides error-checking. Due to these features there are additional fields in the header, compared to UDP, which increase its size and delay. \\
in-order stream protocol

\subsection{UDP}
User datagram protocol (UDP): connectionless, fast, unreliable, simple, provides integrity through checksum of header and payload, send messages (datagrams), stateless protocol

\subsection{SCTP}
for multiple streams, features from both TCP and UDP, characterize SCTP as message-oriented, meaning it transports a sequence of messages (each being a group of bytes), rather than transporting an unbroken stream of bytes as does TCP (https://searchnetworking.techtarget.com/definition/SCTP), 
\\ \\
In the absence of native SCTP, it is possible to tunnel SCTP over UDP. This is called SCTP/UDP and is a protocol that is available in NEAT.

\section{BSD sockets}
General information: 
The Berkeley Sockets,
1983,
Universal standard,
TCP and UDP,
How it works,
Example, 
API docs (socket(), send(), ...),
How it has evolved (or not - why it no longer is good enough) - Old, need to evolve

\section{NEAT}
API, Architecture, Policy, how it works (example?) and why use it. \\
NEAT was first introduced in … and its purpose was to ... 
NEAT is a callback based library that uses libuv as an event library. 
the NEAT project wants to achieve a complete redesign of the way in which Internet applications interact with the network.\\
our goal is to allow network “services” offered to applications – such as reliability, low-delay communication or security – to be dynamically tailored based on application demands, current network conditions, hardware capabilities or local policies, and also to support the integration of new network functionality in an evolutionary fashion, without applications having to be rewritten (source). \\
Use of happy eyeballs algorithm to determine transport protocol /
Transport protocol selection using Happy Eyeballs \\
This is a software library built above the socket API to provide networking applications with a new API offering platformand protocol-independent access to Transport Services (source)


\subsection{Why neat?}
Mention de-ossifying of the internet transport and why NEAT is therefore needed. Other approaches  (similar ways will be mentioned later).

\section{Libuv}
Libuv is a support library originally written for Node.j. It is designed around the event-driven asynchronous I/O model \\
(http://docs.libuv.org/en/v1.x/design.html). 
\\
Began in 2009 (node.j project),
Enforces an asynchronous, event-driven style of programming,
High level of abstraction,
core job is to provide an event loop and callback based notifications of I/O and other activities \\(https://nikhilm.github.io/uvbook/)
\\
Mention:
API docs (uv loop t, ...),
Event loop - used in NEAT,
Two other abstractions: handles (long lived) and requests (short lived),
The I/O (input / output) loop is central
\\
\\
Due to the inflexibility and age of the BSD sockets there have been proposed several other network transport API to replace them. Those similar to NEAT include post sockets (...), taps by IEFT  and network.framework by Apple. These will be discussed in further details next/ below.

\section{Network.framework}
\textit{The next three sections may be in the same section titled something like "Related Solutions/ Libraries/ Alternatives / Approaches ...", where then each of these topics will be in a subsection.}\\ \\
Network.framework is another API developed as an alternative to the BSD sockets. It is developed by Apple and the already established class URLSession, which is used for loading HTTP- and URL-based resources (source), is built on it. This API is also a library like NEAT... 
\textit{(why use it? Security? How is it different from NEAT? Very little information on it...).}
Many of the problems that the BSD sockets have, Network.framework solves. It supports TLS and DTLS by default and handles dual stack cases, as well as only IPv6 networks.
\textit{Discussions with IEFT, the basics of the classes used (listeners, etc) or just that it is similar or not, asynchronous model for reading and writing, in swift or c, ...}

\section{Post sockets}
Post sockets began development in ... and was presented as an alternative /new standard  API to the already existing BSD sockets. \\
Replacement for sock stream abstraction (TCP) with message abstraction. \\
Centered around Message carrier (like socket) - logically group Messages for transmission and reception.\\
Post Sockets describes an abstract API that is intended to be broadly useful for applications, and that can support a wide range of transport protocols and services. The API we propose
is deliberately tightly coupled with applications in a number of places, since many of the protocols of interest also exhibit similarly tight coupling [7].\\
message-oriented API,
Similar to SCTP (based on) \\
Post Sockets is primarily intended to be used with a high-level systems programming languages (e.g., Swift, Go, Rust), rather than as a low-level C API.

\section{Taps IETF}
\textit{IETF chartered a working group called “Transport Services” (TAPS) in September 2014.
Goal:  help application and network stack programmers by describing an (abstract) interface for applications to make use of Transport Services
Define services that are important to appliances
Check if the desired services are available at end points and if not that there is a fall back available}(source).


\section{CPU and memory usage}
May mention something short about this and then move on to measurement options. \\
Tools / methods used for the measurement of CPU time and memory usage (and about what this is and what "types" there is) - valgrind (massif, callgrind), ... \\
May be moved to another chapter instead of being here (experimental chapter?)

\chapter{Experimental}
What are the experiments that will be carried out? Explain what is needed and what has been done. Possbible sections: setup, test cases, teacup, ... \\
Ran the tests first on one computer using localhost; then used PI4 cluster (testbed).
\\
Should also mention what improvements have been done to the NEAT code and what the purpose behind the changes is and then see if they have the desired effect.

\chapter{Evaluation/ Results}
Evaluate what has been found in the experiments carried out in the lab. Did we find the same results that the previous master thesis found? If our results differ, in what way? Are our set up different (the previous master thesis mentioned the json handling used a lot of space, have we solved this?)


\chapter{Conclusion/ Discussion}
The results of the thesis and an answer to the research question. May be discussed as well or just a summary of what we found out.

\chapter{Further work}
Is there anything further that can be done that we figured out could be done? Not necessary if we don't find anything. This may also be a section under the conclusion/ discussion.

\chapter{References}
All the references used - endnote does not work with latex but can be imported into bibTex which is a reference tool that works with latex.\\
\\
This should also include list of figures, tables and such (this may also be in the beginning of the thesis).

\appendix
\chapter{Appendix title}
Any "extra" information.


\end{document}